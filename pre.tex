% Author:Zhuming Shi, Peking University
% Theme from https://github.com/matze/mtheme

\documentclass[12pt,AutoFakeBold]{beamer}
\usepackage[english]{babel}

\usetheme{metropolis}

\usepackage{xeCJK} % Chinese support
\setCJKmainfont{SimSun} % Chinese front
\usepackage{booktabs}%绘制三线表

\author{信息科学技术学院\ 施朱鸣}
\title{转专业交流会}
\date{2019年11月23日}

\begin{document}
    \frame[plain]{\titlepage}
    \frame{\frametitle{Outline}\tableofcontents[hideallsubsections]}
    \section{我的经历}
    \begin{frame}
        \frametitle{我的经历}
        我是谁?我从哪里来?我往那里去?
        \begin{itemize}
            \item 北大18级本科生,化学竞赛保送生,对化学渐渐失去兴趣后决定尝试计算机专业
            \item 化学与分子工程学院18级本科生
            \item 降转到信科计算机科学与技术专业19级本科生
        \end{itemize}
    \end{frame}
    \begin{frame}
        \frametitle{我的经历}
        高数B$81.5$,计概A$89$,线代B$99$。第一学期绩点$3.7$以上。

        第一学期Wonderful了化院专业课普通化学(当事人现在就是很后悔,非常后悔 x)

        第二学期因交通事故受伤缓考了几乎全部课程。
    \end{frame}
    \section{如何转专业}
    \subsection{转专业申请}
    \begin{frame}
        \frametitle{北京大学本科生学籍管理办法(2019年7月)}
        第二十一条\  学生可以申请转院系转专业,但有下列情况的除外:

        (一)教育部有相关规定或者录取前与学校有明确约定不得转专业的;

        (二)依据定向、委托培养协议或规定不能转专业的;

        (三)本科三年级(含)以上学生;

        (四)正在休学或保留学籍的学生;

        (五)已达到退学程度的学生。
    \end{frame}
    \begin{frame}
        \frametitle{北京大学本科生学籍管理办法(2019年7月)}
        第二十二条\ 学生可以在第一学年末或第二学年末申请转院系转专业,各院系根据专业培养规模上限、准入学术要求对申请转入学生进行审核。退役或创业后复学的学生,可在同等条件下优先考虑。

        第二十三条\ 转院系转专业每学年办理一次。转院系转专业学生一般转入同年级,经院系审核确需降级的可降级转入(在校学习时间不得超过六年)。

        医学部与本部互转,还需符合医学部相关管理办法。
    \end{frame}
    \begin{frame}
        \frametitle{转专业的流程}
        教务部在dean上发转专业工作通知(去年4月15日)

        往年包含申请、院系初审、8月中旬教务部终审。

        往年要求转专业学生\alert{务必}在秋季学期正式开学前\alert{两周的工作日}到校办理转专业手续,不建议赶着ddl过来。
    \end{frame}
    \subsection{转专业硬性要求}
    \begin{frame}
        \frametitle{转专业的硬性要求}
        硬性要求和考核方式都会在dean的通知附加文件里面写明,下面列出去年转信科的要求:

        \begin{itemize}
            \item 成绩优良,\alert{绩点≥3.0}的理科院系学生。
            \item 计算机科学与技术、软件工程、数据科学与大数据技术和智能科学与技术等四个专业要求\alert{《计算概论A》 ≥80分或《计算概论(B)》 ≥85分}。
            \item 计算机科学与技术专业“图灵班”和智能科学与技术专业“图灵班”的报名要求:2018级成绩优良,绩点≥3.0的校本部理科院系学生(含元培学院学生),数学A类课程(《数学分析》、《高等代数》)≥80分,且《计算概论A》 ≥85分。
        \end{itemize}
    \end{frame}
    \begin{frame}
        \frametitle{接收人数}
        \begin{center}
            \begin{tabular}{ccc}
                \toprule
                专业&拟接收&实际接收\\
                \midrule
                电子信息科学与技术专业&$≤20$&$6$\\
                电子信息工程专业&$≤20$&$3$\\
                微电子科学与工程专业&$≤20$&$1$\\
                集成电路设计与集成系统专业&$≤20$&$1$\\
                计算机科学与技术专业&$≤20$&$11$\\
                数据科学与大数据技术专业&$≤10$&$9$\\
                软件工程专业&$≤10$&$4$\\
                智能科学与技术专业&$≤20$&$16$\\
                总计&$140$&$51$\\
                \bottomrule
            \end{tabular}
        \end{center} 
    \end{frame}
    \section{我在决定转专业后做了什么}
    \begin{frame}
        \frametitle{获取信息的途径}
        \begin{itemize}
            \item 官网:dean.pku.edu.cn 和eecs.pku.edu.cn
            \item 学长学姐,转专业大群
        \end{itemize}
    \end{frame}
    \begin{frame}
        \frametitle{决定转专业后做些什么?}
        \begin{itemize}
            \item 想清楚为什么要转到那个专业
            \item 选修转入院系基础专业课(\alert{量力而行})
            \item 好好准备期末,成绩不能太低
            \item 有时间可以去接触一下转入院系的科研
        \end{itemize}
    \end{frame}
    \subsection{特殊情况怎么办}
    \begin{frame}
        \frametitle{特殊情况怎么办}
        有缓考怎么办?
        \begin{itemize}
            \item 保留缓考的事由(比如医院证明)
            \item 联系教务说明情况
        \end{itemize}
        有挂科怎么办?
        \begin{itemize}
            \item 最好别挂了吧……
            \item (第二学期)感觉要挂就中期退课
            \item 及时联系教务解释
        \end{itemize}
        今年最终没转成怎么办?
        \begin{itemize}
            \item 想想要不要再战
            \item 自己权衡能否(正常)毕业
        \end{itemize}
    \end{frame}
    \section{转到信科后的日子}
    \begin{frame}
        \frametitle{转到信科后的日子}
        \begin{block}{北大不收敛定律}
            \begin{equation*}
                \begin{split}
                    &\forall a \in PKU, \exists b \in PKU,\\
                    &s.t \ a.skill_1 < b.skill_1 \ and \ a.skill_2 < b.skill_2
                \end{split}
            \end{equation*}
        \end{block}
    \end{frame}
    \begin{frame}
        \frametitle{毕业生去向}
        2018届信科本科生毕业去向(数据来自北京大学学生就业指导服务中心)
        
        \begin{center}
            \begin{tabular}{cc}
                \toprule
                去向&比例\\
                \midrule
                国内升学&47.7\%\\
                出国(境)深造&34.2\%\\
                灵活就业&15.4\%\\
                \bottomrule
            \end{tabular}
        \end{center}

        国内升学的142人中,89\%选择留在北大,其中82\%留在信科。
        
        出国(境)深造的99人中,82人选择美国,其中卡耐基梅隆大学20人,哥伦比亚大学7人,加州大学洛杉矶分校6人,76人去往北美top50。

    \end{frame}
    \section{结语}
    \begin{frame}
        \frametitle{结语}
        我的联系方式:shizhuming@pku.edu.cn

        祝大家期末顺利,都能去到适合自己的院系和专业,谢谢!
    \end{frame}
\end{document}